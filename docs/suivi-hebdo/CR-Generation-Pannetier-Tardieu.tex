
\documentclass[11pt]{article}

\usepackage[utf8]{inputenc}
\usepackage[T1]{fontenc}
\usepackage[frenchb]{babel}
\usepackage{amssymb,amsmath}
\usepackage{float,subcaption}
\usepackage[dvipsnames]{xcolor}
\usepackage{bm}
\usepackage[ruled,vlined, french, onelanguage]{algorithm2e}
\usepackage[hidelinks]{hyperref}
\usepackage{textcomp}
\usepackage{esint}
\usepackage{dsfont}
\usepackage{fontawesome}
\usepackage[sc]{mathpazo}
\linespread{1.05}

\usepackage{color}

\definecolor{mygreen}{rgb}{0,0.6,0}
% \definecolor{myblue}{rgb}{0.164,0.293,0.484}
\definecolor{myred}{HTML}{9e1030}
\definecolor{myblue}{HTML}{00539C}
\definecolor{myorange}{HTML}{E08119}
\definecolor{mygray}{HTML}{808080}
\definecolor{mymauve}{rgb}{0.58,0,0.82}
\definecolor{rose}{HTML}{F7CAC9}
\definecolor{serenity}{HTML}{92A8D1}
\definecolor{mycustomgreen}{HTML}{b9d095}

\textwidth=170mm
\textheight=240mm
\voffset=-30mm
\hoffset=-21mm

\newenvironment{changemargin}[2]{\begin{list}{}{%
			\setlength{\topsep}{0pt}%
			\setlength{\leftmargin}{0pt}%
			\setlength{\rightmargin}{0pt}%
			\setlength{\listparindent}{\parindent}%
			\setlength{\itemindent}{\parindent}%
			\setlength{\parsep}{0pt plus 1pt}%
			\addtolength{\leftmargin}{#1}%
			\addtolength{\rightmargin}{#2}%
		}\item }{\end{list}}

\usepackage{array,multirow,makecell}
\newcolumntype{C}[1]{>{\centering\arraybackslash }b{#1}}

\usepackage{graphicx}

\def \bs {\backslash}
\def \eps {\varepsilon}
\def \p {\varphi}

\def \un {\mathds{1}}
\def \RR {\mathbb{R}}
\def \EE {\mathbb{E}}
\def \CC {\mathbb{C}}
\def \KK {\mathbb{K}}
\def \NN {\mathbb{N}}
\def \PP {\mathbb{P}}
\def \TT {\mathbb{T}}
\def \ZZ {\mathbb{Z}}
\def \sA {\mathcal{A}}
\def \sC {\mathcal{C}}
\def \sF {\mathcal{F}}
\def \sM {\mathcal{M}}
\def \sO {\mathcal{O}}
\def \sG {\mathcal{G}}
\def \sE {\mathcal{E}}
\def \sB {\mathcal{B}}
\def \sS {\mathcal{S}}
\def \sT {\mathcal{T}}
\def \sD {\mathcal{D}}
\def \sH {\mathcal{H}}
\def \sN {\mathcal{N}}
\def \sP {\mathcal{P}}
\def \sQ {\mathcal{Q}}
\def \sU {\mathcal{U}}
\def \sV {\mathcal{V}}
\def \sL {\mathcal{L}}
\def \disp {\displaystyle}
\def \vn {\vspace{3 mm} \noindent}
\def \ni {\noindent}
\def \red {\textcolor{Red}}
\def \blue {\textcolor{Blue}}
\def \green {\textcolor{OliveGreen}}
\def \brown {\textcolor{RawSienna}}
\def \purple {\textcolor{Purple}}
\def \navy {\textcolor{NavyBlue}}
\def \orange {\textcolor{Orange}}
\def \pink {\textcolor{Magenta}}
\def \d {{\rm d}}
\def \div {\nabla \cdot}
\def \rot {\nabla \times}
\def \grad {\nabla}
\def \lap {\Delta}
\def \vit {\bm{u}}
\def \grav {\bm{g}}
\newcommand{\peq}{\mathrel{+}=}
\newcommand{\meq}{\mathrel{-}=}
\newcommand{\feq}{\mathrel{*}=}

\def \HRule {\rule{\linewidth}{0.5mm}}


\begin{document}

\noindent
Techniques de Maillage, 2020/2021 \hfill Valentin PANNETIER, Alexis TARDIEU

\vspace{5mm}

\begin{center}
	\Large{\textbf{Génération de maillage 2D de type Delaunay}} \\[2mm]
	\Large{\textit{Compte-rendu de l'avancement du projet}}
\end{center}

\vspace{5mm}

\tableofcontents
\vn


\section{Date : 4 décembre 2020}
%\hrule\vspace{3mm} 
\noindent
\textbf{Plan prévisionnel}

\vn
$\bullet$ Création de la structure de stockages, facilitant l'insertion d'arêtes et de swap --> Il faut que ce soit facile à changer.

\vn
$\bullet$ Lecture et écriture de fichiers (par exemple en entrée un .mesh 1D plongé dans du 2D, en sortie un .mesh classique)


\vn
$\bullet$ Construction de la partie : noyau de Delaunay

\begin{enumerate}
	\item Localisation du point de triangulation -> Simple entrée avec IN/OUT du triangle
	\item Identification de la cavité de Delaunay
	\item Suppression et reconnection de la cavité
\end{enumerate}

\vn
$\bullet$ Créer les swap d'arêtes (listes et étude de la façon de faire)

\vn
$\bullet$ Trouver la boite englobante --> Facile ? (intuitif min-max \& eps, meilleure boite ? Envelope convexe ?)

\vn
$\bullet$ Itérer le noyau pour tous les points de discrétisation 

\vn
$\bullet$ Forcer la frontière

\vn
$\bullet$ Qualité des éléments -> Subdiviser pour améliorer la qualité ? À voir !

\vn
$\bullet$ À partir du nombre de points souhaités, subdiviser les plus gros éléments avec le respect de la discrétisation initiale ! Attention aux proportions de deux triangles côte-à-côte --> Critère !

\vn
$\bullet$ Points d'optimisation

\begin{enumerate}
	\item Accès à la donnée ?
	\item Détecter si oui ou non on est dans la cavité de Delaunay ?
	\item Identifier les éléments les + "mauvais" ?
	\item Déplacer des points de discrétisation ? Peut-être trop dense ou pas assez ?
	\item Attention au rapport de taille ! -> Mauvaise solution numérique
\end{enumerate}

\vn	
$\bullet$ Autres idées : 

\begin{enumerate}
	\item Création de carte de taille, à partir de la densité de discrétisation du(des) bord(s) ? Et donc subdivision par rapport à cette carte -> vers unit mesh
	\item Mailler deux domaines qui se touchent, avec partage de points de bord ?
	\item Tester sur les différents maillages sur le même problème -> Étude de l'erreur en fonction du maillage et des critères (donc nécessité d'être configurable)
	\item P2, P3 etc ? (Option car des notions de courbures interviennent)
\end{enumerate}

\vn
$\bullet$ Cas tests

\begin{enumerate}
	\item tore ? (trou)	
	\item étoile ? (arêtes vives et coins ?)
	\item imposition d'une discrétisation interne --> aile d'avion ?
	\item Forme de pièce industrie ? Avec des zones bcp plus denses ?
\end{enumerate}



%\begin{align*}
%	\left(x_a - x_s\right)^2 + \left(y_a - y_s\right)^2 &= r^2\\
%	\left(x_b - x_s\right)^2 + \left(y_b - y_s\right)^2 &= r^2\\	
%	\left(x_c - x_s\right)^2 + \left(y_c - y_s\right)^2 &= r^2\\
%\end{align*}


\begin{table}[H]
	\centering
	\begin{tabular}{l|l}
	\hline	
	\textbf{Valentin} & \textbf{Alexis} \\
	\hline
		\begin{tabular}{l}
			Création du répertoire \textcolor{mygreen}{\faCheck}\\
			Classe Cell \& Mesh \textcolor{mygreen}{\faCheck}\\
			Écriture \textcolor{mygreen}{\faCheck}\\
			Création des maillages d'entrées \textcolor{mygreen}{\faCheck}\\
			Forcer la frontière, et carte de couleur\\
			Carte rapport de taille ?\\
		\end{tabular}
		& 
		\begin{tabular}{l}
			Classe Point \textcolor{mygreen}{\faCheck}\\
			Lecture d'un .mesh \textcolor{mygreen}{\faCheck}\\
			Swap\\
			Qualité des éléments \textcolor{mygreen}{\faCheck}\\
			Boite englobante\\
%			Solveur (Utilisation d'Eigen ?)\\\textsl{}
		\end{tabular}\\
	\hline
	\end{tabular}
	\caption{Répartition}
\end{table}






\end{document}